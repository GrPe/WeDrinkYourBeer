\documentclass[12pt]{article}

\usepackage[T1]{fontenc}
\usepackage[polish]{babel}
\usepackage[utf8]{inputenc}
\usepackage{lmodern}
\selectlanguage{polish}

\title{%
	We Drink Your Beer\\
	\large Game Design Document}
\author{Grzegorz Perzanowski}

\begin{document}
	\maketitle
	\newpage
	\tableofcontents
	\newpage
	
\section{Overview}
\subsection{Concept}
	Gra na Androida polegająca na obronie browaru przez falami studentów.
\subsection{Gatunek}
	Tower Defence 
	
\newpage
\section{Gameplay}
	Celem gracza jest ochrona browaru przed falami przeciwników (studentów). Gracz ma możliwość
	budowania wieżyczek obronnych na wyznaczonych polach za walutę zdobywaną w trakcie gry.
\subsection{Czynności wykonywane przez gracza}
\begin{itemize}
	\item Budowa wieżyczek
	\item Ulepszanie Wieżyczek
\end{itemize}
\subsection{Czynności wykonywane automatycznie}
\begin{itemize}
	\item Zbieranie zasobów
\end{itemize}
	
\newpage
\section{Mechanics}
\subsection{Przeciwnicy}
	Poruszają się po wyznaczonych ścieżkach w kierunku browaru.
\subsection{Wieżyczki}
	Stawiane przez gracza. Odpowiedzialne są za zadawanie obrażeń przeciwnikom.
\paragraph{Single}
	Wieżyczka strzelająca ogniem pojedynczym. Średnia szybkość i średnie obrażenia.
\paragraph{Stream}
	Wieżyczka strzelająca ogniem ciągłym. Stałe lecz niskie obrażenia.
\paragraph{Cannon}
	Wieżyczka strzelająca z niską częstotliwością, zadająca duże obrażenia.
\paragraph{Side Effects}
	Każda z wieżyczek po ulepszeniu może nakładać na przeciwników negatywne efekty.
\subsection{Maps}
	Prostokątna. Jedna lub więcej ścieżek biegnących od krawędzi mapy do browaru.

\newpage
\section{Lore}
\subsection{Story}
\subsection{Towers}
\subsection{Enemies}

\newpage
\section{Interface}
\subsection{Main Menu}
\subsection{Wybór misji}
\subsection{Credits}
\subsection{User Interface}

\newpage
\section{Game Controls}
Tylko poprzez dotyk - pojedyncze tapnięcie 

\newpage
\section{Technicals}
\subsection{Target}
Android 5.0
\subsection{Development Software}
	\begin{itemize}
	\item Android Studio + LibGDX
	\item GraphicsGale
	\item Additional Package: FreeTypeFontGenerator
	\end{itemize}


\newpage
\section{Game Art}
\subsection{Overview}
Rzut z lotu ptaka / izometryczny. Trochę na wzór starych pokemonów. PixelArt

\section{Audio}
Wiadomo tylko tyle, że ma być.

\newpage

\section{Technical Documentation}
\subsection{State Machine - States}
\paragraph{Enemy}
\begin{itemize}
	\item Walking
	\item Dying
\end{itemize}

\paragraph{Player}
\begin{itemize}
	\item Regular
	\item Build
	\item FailMission
	\item SuccessMission
\end{itemize}

\paragraph{Tower}
\begin{itemize}
	\item Waiting for enemy
	\item Attack
	\item building in progress
	\item upgrade in progress
\end{itemize}

\subsection{State Machine - Transition}
\paragraph{Enemy}
\paragraph{Player}
\paragraph{Tower}

\end{document}